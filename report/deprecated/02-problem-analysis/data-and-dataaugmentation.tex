\section{Data}\label{sec:data}
This project uses \gls{mnist} which is descibed by its creators as
\textquote{The MNIST database of handwritten digits, available from this page, has a training set of 60,000 examples, and a test set of 10,000 examples. It is a subset of a larger set available from NIST. The digits have been size-normalized and centered in a fixed-size image.}\todo{Missing citations}


We chose it because its intresting becuause its cool to try and work a real problem which could be used very practically for text recognition. Another reason is it's very used so it easy to compare ouselves to other projects using this dataset. Which makes us able to get a better start on a topic which were exploring for the first time. We also want somthing simple so we can better see the effects of FE. We think we can get more general learning when we dont have to worry about the exploratory part of working with a dataset. This is due to mnist being already explored by other reserchers to a high degree. Another reason we choose mnist is that the classes are evenly distributed Which makes it easier to work with in some aspects. But it might be too regular to really show \gls{fe} so we will augment it so we can explore it in a controlled manner.
We have chosen to achieve this using dataaugmentaion which is defined as

\textquote{Data augmentation in data analysis are techniques used to increase the amount of data by adding slightly modified copies of already existing data or newly created synthetic data from existing data. It acts as a regularizer and helps reduce overfitting when training a machine learning model. It is closely related to oversampling in data analysis.}
\todo{Missing citations}

We also chose to augment the data so its possible to show how the \gls{fe} and Model handle different shapes of data while still having a baseline. This helps us make more general inferences and ituition about \gls{fe} and ML it will in particular show which \gls{fe} Methods handle different data shapes the best. it will also help show how augmentation in general can effect performance models  and FE.
We Chose the following list of augmentations.  These were chosen because they are well suited for images since they create variations of the basic Images. Options like rotations also make it possible to really challenge our \gls{fe} and find out what they can do. Another thing we would like to try is removing parts of the rotation and seeing how this affects the model.

\begin{itemize}
    \item rotation
    \item shift
    \item removing pixels
    \item scaling
    \item sharpening
    \item blurring
\end{itemize}

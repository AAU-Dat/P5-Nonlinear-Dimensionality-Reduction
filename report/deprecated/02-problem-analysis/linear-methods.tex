\section{Linear methods}\label{sec:linear-methods}
This section will present the chosen linear methods that will be used in the project. The reason why linear methods were chosen is because they are a relevant for the analysis of the high-dimensional nature of image data \cite{linear-dimensionality-reduction-insights}.

It should be noted, that some of the linear and non-linear methods usually try to solve an optimization problem, in which the function has the form of


\begin{align}
  \phi(Y) = \frac{Y \top AY}{Y \top BY},
\end{align}


which can be optimized by using eigendecomposition \cite{dimensionality-reduction-comparative-review}. Some of these methods performing eigendecomposition rely on a full matrix which includes the covariances between datapoints. A covariance matrix $S_y$, which contains the variance between each pair of elements, can be aquired as follows: (The example is inspired from \gls{pca}, which will be presented in Subsection \ref{subsec:principal-components-analysis})


\begin{align}
  S_y = \frac{1}{N}\sum\limits_{i=1}^{N}(y_i - \mu)(y_i - \mu)\top, where
\end{align}


\begin{align}
  \mu = \frac{1}{N}\sum\limits_{i=1}^{N}x_i,
\end{align}


$\mu$ denotes the average value for each column. $N$ denotes the number of data points. $y_i$ denotes the variance for the embedded data.

Hence, after calculating the covariance matrix for $\mathbf{X}$, it can form the linear mapping by applying eigendecomposition on


\begin{align}
  cov(X)M = \lambda M, where
\end{align}


$\lambda$ represents eigenvalues with $d$ eigenvectors(also called principal components in \gls{pca}), which are chosen. Thus mapping the high-dimensional data onto lower dimensions can be written as $\mathbf{Y=XM}$, where $\mathbf{X}$ is the input matrix, $\mathbf{M}$ is the transformation matrix, and $\mathbf{Y}$ is the embedded output.

\subsection{Principal Components Analysis}\label{subsec:principal-components-analysis}
\gls{pca} tries to map the high-dimensional data so that it maximizes the amount of variance in the embedded data. By maximizing the variance it can be expressed as maximizing the cost function C($\mathbf{Y}$) $tr(M\top cov(X)M)$, where cov($\mathbf{X}$) is the covariance matrix for $\mathbf{X}$ \cite{dimensionality-reduction-comparative-review}.

\subsection{Factor Analysis}\label{subsec:factor-analysis}
\gls{fa} tries to correlate the input variables into fewer latent factors, or variables, where each variable supplies parto fht weight of the factor. It can be noted that the relation between the \gls{fa} family and \gls{pca} is that \gls{pca} is a type of \gls{fa}, specifically \gls{efa}  contains a type, called \gls{pfa}.

The difference between the \gls{pfa} and \gls{pca} is that \gls{pfa} assume that there are unobserved data that need to be be extracted, while \gls{pca} extracts meaning from the data itself \cite{factor-analysis-introduction}. Another difference is that FPA distinguishes between common variance and unique variance -- common variance is applies to data that are highly correlated, and thus share a lot of variance, whereas unique variance is applies to data that are not common -- whereas \gls{pca} assumes that there is only common variance.

\subsection{Non-negative matrix factorization }\label{subsec:non-negative-matrix-factorization}
\gls{nmf} tries to factorize the input matrix $X$ into two matrices $WH$ such that $WH ~ A$. This means that \gls{nmf} tries to minimize the difference between the $X$ and $WH$ \cite{non-negative-matrix-factorization}. However, computing \gls{nmf} can be considered computationally expensive.

\subsection{Linear Discriminant Analysis}\label{subsec:linear-discriminant-analysis}
\gls{lda} is trying construct a lower dimensional space which maximizes the ratio between \textbf{between-class} variance and \textbf{within-class} variance, which maximizes class separability. By between-class it is meant the distance between means of different classes, and within-class as the distance between the mean and the observations of each class. In one equation,


\begin{align}
  LDA = \frac{Variance_{BetweenClass}}{Variance_{WithinClass}}
\end{align}


\gls{lda} maximizes the separability between classes and minimizes the separability within classes so as to get the largest value possible.


\gls{lda} can be divided into two types:

\gls{ci} and \gls{cd}.

\gls{ci} computes one global transformation matrix for the data, whereas \gls{cd} takes a more local approach; it computes the transformation matrix for each class, along with the eigenvectors and eigenvalues \cite{linear-discriminant-analysis-tutorial}.



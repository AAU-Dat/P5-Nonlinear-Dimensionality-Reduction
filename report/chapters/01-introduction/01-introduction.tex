\chapter{Introduction}\label{cha:introduction}
\todo[inline]{Chapter should probably contain: The initial problem (If we have one), motivation, and the scope or background of the project or theme. Report outline at the end.}
\noindent
In this project we will study some common methods of dimensionality reduction. Inspired by the MNIST dataset for digit recognition.

\paragraph{Keywords:} dimensionality reduction, linear methods, nonlinear methods, MNIST, Computer Vision (CV), Machine Learning, machine intelligence (artificial intelligence).

Use sources \cite{IBM-machine-intelligence} and \cite{IBM-computer-vision} for superficial overview and explanations of umbrella terms.

% Might be relevant to us: MNIST and fashion-MNIST (https://github.com/zalandoresearch/fashion-mnist).

\subsection*{On theory driven projects}
\blockcquote{Projectmodule}{The overall purpose of the project module is for the student to acquire the ability to analyze and evaluate the application of methods and techniques within database systems and / or machine intelligence to solve a specific problem. \textbf{This includes analyzes of the formal properties of the techniques and an assessment of these properties in relation to any requirements for the solution to the specific problem}. [...]

In this project module, the project work is primarily driven by theoretical and analytical considerations about the methods and techniques used. For a specific problem area, a project could, for example, be based on specific performance requirements for the developed software solution, and the project work can thus be guided by the solution's algorithmic time / space complexity as well as formal analyzes and considerations of its theoretical properties and performance guarantees.}

\section{Motivation}\label{sec:motivation}
High-dimensional data (i.e.\ data that requires more than three dimensions to be represented) can often be difficult to work with. Not only is it difficult to interpret and visualize but also can require a high use of computational resources. For these (and many more) reasons, it is important to study dimensionality reduction methods. These methods are usually used in exploratory data analysis and for visualization purposes.

The most usual methods of dimensionality reduction are \textbf{linear methods}. These methods might assume that the features in the original data are independent and they can produce reduced data by a linear combination of the original data. These assumptions might not apply to all datasets. In fact, there are cases in which linear methods do not capture important features of a dataset. For these cases one can use \textbf{nonlinear methods}. These methods can be used for more general cases while preserving important information from data.
\section{Report outline}\label{sec:report-outline}
The report is structured as follows:

\begin{itemize}
  \item \textbf{Introduction} - This chapter
  \item \textbf{Problem Analysis} - Chapter \ref{cha:problem-analysis}
  \item \textbf{Theory} - Chapter \ref{cha:theory}
  \item \textbf{Methodology} - Chapter \ref{cha:methodology}
  \item \textbf{Results} - Chapter \ref{cha:results}
  \item \textbf{Discussion} - Chapter \ref{cha:discussion}
  \item \textbf{Conclusion} - Chapter \ref{cha:conclusion}
\end{itemize}

The introduction describes the initial problem and the motivation for the project.

The Problem Analysis chapter dives into the initial problem and leads to a final problem statement.

The Methodology chapter describes the methods and theory used to explore the problem statement. It also describes the data used and how it was collected/created.

The Results chapter is an evaluation of the results of the project.

The Discussion chapter is a discussion of the results and the project as a whole.

The Conclusion chapter provides a summary of the project and the results. It also provides perspective and reflection on the project and the process.

\section{Motivation}\label{sec:motivation}
High-dimensional data (i.e.\ data that requires more than three dimensions to be represented) can often be difficult to work with. Not only is it difficult to interpret and visualize but also can require a high use of computational resources. For these (and many more) reasons, it is important to study dimensionality reduction methods. These methods are usually used in exploratory data analysis and for visualization purposes.

The most usual methods of dimensionality reduction are \textbf{linear methods}. These methods might assume that the features in the original data are independent and they can produce reduced data by a linear combination of the original data. These assumptions might not apply to all datasets. In fact, there are cases in which linear methods do not capture important features of a dataset. For these cases one can use \textbf{nonlinear methods}. These methods can be used for more general cases while preserving important information from data.
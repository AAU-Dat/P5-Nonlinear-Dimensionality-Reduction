\section{Problem Statement}\label{sec:problem-statement}
\subsection{Audio} 
For models that predict what music is popular or what genre the music is we would like to see how big of an effect feature engineering has for the model. We would like to investigate which kind of dimensionality reduction works best considering both linear and nonlinear aproaches and what they contribute to in the model and when it is a better fit. The performance of these dimensionality reductions is evaluated based on how they affect the performance of the model and their visualisations.

\subsection{Pokemon}
For a model that clasifies Pokemon we would like to see how big of an effect feature engineering has for the model. We would also like to investigate which kind of dimensionality reduction works best and consider both linear and nonlinear aproaches and what they each contribute and when theyre correct to use. The performance of these aproaches might be evaluated based on their visualisations and how they affect model performance.

<<<<<<< HEAD:report/chapters/Problem_statement
\section{Match data}
For models that predict the outcome of football matches we would like to see how big the effect of feature engineering has for the model. We would also like to investigate which kind of dimensionality reduction works best considering both linear and nonlinear aproaches and what they contribute to in the model and when it is a better fit. The performance of these dimensionality reductions is evaluated based on how they affect the performance of the model and their visualisations.

\section{??}
The problem is clasification of Pokemon types. This will be done by using machine intelligence models to predict the type of a pokemon based on its colours. The 2d-images will be pre-proccesed to have the same size and clasified based on their RGB colours values, using both linear and nonlinear dimensionality reduction methods. The metrics of the evaluation will be speed as images contains large amount of data. 

Clasification on pokemons, there are 3 different colours for each type of pokemon a combination of RGB


predict color in grey scale based on the color of the pokemon
=======
\subsection{Match data}
For models that predict the outcome of football matches we would like to see how big the effect of feature engineering has for the model. We would also like to investigate which kind of dimensionality reduction works best considering both linear and nonlinear aproaches and what they contribute to in the model and when it is a better fit. The performance of these dimensionality reductions is evaluated based on how they affect the performance of the model and their visualisations.
>>>>>>> 3dc2fe24afccafc0b8c353df8cee3d5e9124c5e1:report/chapters/02-problem-analysis/problem-statement.tex

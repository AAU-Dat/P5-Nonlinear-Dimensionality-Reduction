\section{Problem statement}\label{sec:problem-statement}

\emph{This project aims to explore the impact of dimensionality reduction, comparing linear and nonlinear dimensionality reduction techniques. This will be done in regards to the performance of a machine learning model, for the problem of image classification and recognition, of the \gls{mnist} dataset.}

%\emph{This project explores the impact of data preprocessing on the performance of a \gls{svm} machine learning model, for the computer vision problem of image classification and recognition. By data preprocessing is meant dimensionality reduction on augmented data, comparing linear and non-linear dimensionality reduction techniques. 
%The machine learning model is trained on the dimensionality reduced data and the performance is evaluated using accuracy, precision, recall, and F1 score. A \gls{cnn} model is also trained on the same data for a discussion on the tradeoff between the evaluation metrics, speed, and explainability. The data used is the \gls{mnist} database.}

%\emph{This project explores the impact of data preprocessing on the performance of a machine learning model, for the computer vision problem of image classification and recognition. By data preprocessing is meant dimensionality reduction on augmented data, comparing linear and non-linear dimensionality reduction techniques. The machine learning model is trained on the dimensionality reduced data and the performance is evaluated.}


%What impact does the different dimensionality reduction techniques have on the relevant metrics of the model? Where can this impact become beneficial and where can it cause issues for the model? What are the reasons for augmentation of data? In regards to evaluation of the model, what metrics are most important to consider? What is the tradeoff between the metrics, speed, and explainability of the model? These are some of the questions that will be answered in this project.


%Linear and non-linear dimensionality reduction techniques are compared, to gain insight into the benefits of each. The machine learning model will be evaluated on metrics, to determine the performance of the model, to gain a clear understanding of the gains and consequences of dimensionality reduction techniques and augmentation of data. 

%Why would we want to dimensionality reduce data?
%Why would we want to augment data?
%Consequences of dimensionality reduction and augmentation on the data.
%What metrics are used to evaluate the model?

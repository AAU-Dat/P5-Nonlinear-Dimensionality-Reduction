\section{Machine learning}\label{sec:machine-learning}
Figure~\ref{fig:python-pipeline-model} illustrates a general \gls{ml} pipeline.

The pipeline consists of four main steps: data, \gls{fe}, \gls{ml} model training, and model evaluation. The \gls{fe} step is used to transform the data into a form that is more suitable for the \gls{ml} model. The \gls{ml} model training step is used to train the \gls{ml} model on the data. The model evaluation step is used to evaluate the performance of the \gls{ml} model. The \gls{ml} model can then be used to make predictions on new data.


\begin{figure}[htb!]
    \centering
    \begin{tikzpicture}
        \node (b) [state] {feature engineering};
        \node (c) [state, shift={($(b.east)+(2cm,0)$)}] {model};
        \node (a) [state, shift={($(b.west)+(-2cm,0)$)}] {data};
        \node (d) [state, shift={($(c.east)+(2cm,0)$)}] {evaluation};
        \node (e) [state, shift={($(b.south)+(0,-2cm)$)}] {parameters};

        \draw[arrow, ->] (a) -- node[above,scale=.70,align=center,] {} (b);
        \draw[arrow, ->] (b) -- node[above,scale=.70,align=center,] {} (c);
        \draw[arrow, ->] (c) -- node[above,scale=.70,align=center,] {} (d);
        \draw[arrow, ->] (e) -- node[above,scale=.70,align=center,] {} (c);

        \draw[arrow, ->] (d.north) -- ++(0,0.75) -| (b);
        \draw[arrow, ->] (d.south) -- ++(0,-0.75) -| (c);
    \end{tikzpicture}
    \caption{Simplified machine learning pipeline}
    \label{fig:basic-machine-learning-pipeline}
\end{figure}


However, the data used for a \gls{ml} models must be of high quality, and the models must be able to generalize well to new data. This is where \gls{fe} and dimensionality reduction comes into play. \gls{fe} is the process of transforming data into features that are more suitable for \gls{ml} models. Dimensionality reduction is the process of reducing the number of features in a dataset. This is done to reduce the complexity of the data, and to make it easier to visualize.
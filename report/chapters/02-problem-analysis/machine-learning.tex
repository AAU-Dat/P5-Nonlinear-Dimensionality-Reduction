\section{Machine learning}\label{sec:machine-learning}
A pipeline is good for comparing and contrasting linear and nonlinear methods through the lens of a model efficiently. In this section, there will be a brief Introduction to pipelines, their use, and how to introduce dimensionality reduction to them.

\autoref{fig:basic-machine-learning-pipeline} illustrates a general \gls{ml} pipeline. A \gls{ml} pipeline consists of four main steps: data, \gls{fe}, \gls{ml} model training, and model evaluation. The first step is the data step, which is the data collection. The \gls{fe} step transforms the data into a more suitable form for the \gls{ml} model. The \gls{ml} model training step trains the \gls{ml} model on the data. The model evaluation step is used to evaluate the performance of the \gls{ml} model. The \gls{ml} model can then be used to make predictions on new data~\cite{machine-learning-pipeline-architecture}.

\begin{figure}[htb!]
    \centering
    \begin{tikzpicture}
    \node (b) [state] {feature engineering};
    \node (c) [state, shift={($(b.east)+(2cm,0)$)}] {model};
    \node (a) [state, shift={($(b.west)+(-2cm,0)$)}] {data};
    \node (d) [state, shift={($(c.east)+(2cm,0)$)}] {evaluation};
    \node (e) [state, shift={($(b.south)+(4.92cm,-1.5cm)$)}] {parameters};

    \node (o) [shift={($(c.south)+(0cm,0.13cm)$)}] {};
    \node (p) [shift={($(d.south)+(0cm,0.13cm)$)}] {};

    \draw[arrow, ->] (a) -- node[above,scale=.70,align=center,] {} (b);
    \draw[arrow, ->] (b) -- node[above,scale=.70,align=center,] {} (c);
    \draw[arrow, ->] (c) -- node[above,scale=.70,align=center,] {} (d);
    \draw[arrow, ->] (e) -| node[above,scale=.70,align=center,] {} (o);
    \draw[arrow, <-] (e) -| node[above,scale=.70,align=center,] {} (p);

    \draw[arrow, ->] (d.north) -- ++(0,0.75) -| (b);
\end{tikzpicture}
    \caption{Simplified machine learning pipeline}
    \label{fig:basic-machine-learning-pipeline}
\end{figure}

The \gls{fe} step is where dimensionality reduction is relevant - this step reduces the dimensions to increase the model's performance.

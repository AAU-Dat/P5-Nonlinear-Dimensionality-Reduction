\section{Machine learning models}\label{sec:machine-learning-models}
This section will cover the machine learning models considered for the project. Briefly describing the intuituin behind them, how they work and some of their pros and cons. These machine learning models, can be split into two different categories, being Discriminative and Generative models. Discriminative models are used for classification or Regression problems, and generative models are used for giving the probability of something happening. An example of a model that could be used, is "convolutional neural networks", that is often used for computer vision and image classification, and is one of the most accurate models for the MNIST dataset\cite{MnistStatictics}. But there are also other types of models that do not rely on neural networks, such as Logistic regression, K-nearest neighbors and others.

\subsection{Logistic regression}
Logistic regression is a type of statistical model, that is used for classification and predictive analytics. It estimates the probability of something given some data, the value of this probability is either 0 or 1. An example of this would be a model to determine whether or not a patient has some disease. In this case 0 could mean the model predicts that the patient does not have said disease and 1 could indicate that the model predicts that the patient does have the disease. Logistic regression is used in machine learning and is known as a supervised machine learning model.\cite{WhatIsLogisticRegression} (Supervised machine learning models are defined as models where the data they use is labeled\cite{SupervisedLearning}.)

Logistic regression, has some advantages and disadvantages, such as it being a simple model that it easy to understand, and performing well on linearly separable data. It also does not often overfit, unless the datasets are high in dimensionality. Some disadvantages of logistic regression is that it is not able to handle non-linear data, and it is not able to handle high dimensional data. \cite{LogisticRegressionProsAndCons}

%https://medium.datadriveninvestor.com/logistic-regression-essential-things-to-know-a4fe0bb8d10a
%https://www.ibm.com/topics/logistic-regression
%https://www.ibm.com/cloud/learn/supervised-learning

\subsection{K-nearest neighbors}
K-nearest neighbors is another type of supervised machine learning model. The goal of the algorithm is to find the k nearest neighbors of a data point and then classify the data point based on the majority of the neighbors. The algorithm is used for classification or regression problems, but most often used for classification.\cite{K-NearestNeighborsIBM}

KNN, as many other models for machine learning, has its advatages and disadvantages. KNN is a simple algorithm, and is easy to understand and implement. But it is also not great for scaling, it is known as what is called a "Lazy algorithm", because it uses a lot of memory and datastorage, when compared to other models, meaning that time and cost efficiency can take a hit at larger datasets. Also the model is also prone to overfitting, depending on the value of K, with a low value often leading to overfitting and a high value often leading to underfitting. \cite{K-NearestNeighborsIBM}
%https://www.ibm.com/topics/knn#:~:text=The%20k%2Dnearest%20neighbors%20algorithm%2C%20also%20known%20as%20KNN%20or,of%20an%20individual%20data%20point.

\subsection{Support vector machines}
Support vector machine, also called SVM, works by finding a hyperplane that separates the datapoints into two classes. The hyperplane is found by maximizing the margin between the two classes. The goal of the model is to find the hyperplane that has the largest margin between the two classes. (largest margin meaning the 'widest' hyperplane that still seperates two classes') The model is used for classification problems, and is a supervised machine learning model. Hyperplanes help classify datapoints, where depending on where the datapoint is located in relation to the hyperplane, it can be classified as either one of the two classes. These hyperplanes are easy to visualize in 2d and 3d, but becomes more difficult to visualize in higher dimensions, but the model still works in higher dimensions.\cite{SupportVectorMachines}

SVM, has some advantages and disadvantages. The model is very good at handeling high dimensional data and has good accuracy. Especially when the number of dimensions exceeds the number of samples, the SVM is proven to be useful.
But it does not performe well when handling large datasets, because of long training periods, and it is also not very good at handling noisy datasets, as it is sensitive to data overlap in target classes.\cite{SVMProsAndCons}
%https://towardsdatascience.com/support-vector-machine-introduction-to-machine-learning-algorithms-934a444fca47
%https://roboticsbiz.com/pros-and-cons-of-support-vector-machine-svm/


\subsection{Convolutional neural networks}
Unlike previously mentioned models, convolutional neural networks is a deep learning algorithm, that is used for image classification. Compared to the other models mentioned, CNN has the ability to learn the characteristics of the images, and then use these characteristics to classify the images. There are three different types of layers in a CNN, the convolutional layer, the pooling layer and the fully connected layer. The convolutional layer is used to extract features from the images, the pooling layer is used to reduce the dimensionality of the data, and the fully connected layer is used to classify the images.\cite{CNNIBM}

CNN is a very accurate model, as mentioned at the beginninng of this section, CNN is the model used to acchive one of the highest accuracy ratings on the Mnist data set. Another reason CNN is a strong model, is because it can learn what characteristics are important for the classification of the images. But CNN also has some disadvantages, such as it being a very complex model, it being very computationally expensive and the need for large datasets.\cite{CNNProsAndCons}


%https://www.geeksforgeeks.org/difference-between-ann-cnn-and-rnn/
%https://www.ibm.com/cloud/learn/convolutional-neural-networks
%https://towardsdatascience.com/a-comprehensive-guide-to-convolutional-neural-networks-the-eli5-way-3bd2b1164a53

%Notes
%CNN(Convolutional Neural Networks) is often used for computer vision, and is so far the model that has acchived one of the highest accuracy scores for the Mnist dataset, according to https://paperswithcode.com/sota/image-classification-on-mnist.
%Support vector machine

%Sources:
%https://fullscale.io/blog/machine-learning-computer-vision/
%https://www.sciencedirect.com/science/article/pii/S1877050920308218
%https://machinelearningmastery.com/how-to-develop-a-convolutional-neural-network-from-scratch-for-mnist-handwritten-digit-classification/
%https://www.milindsoorya.com/blog/handwritten-digits-classification
%https://arxiv.org/abs/2107.00436
%https://link.springer.com/chapter/10.1007/978-981-13-3441-2_11
%https://www.bhu.ac.in/research_pub/jsr/Volumes/JSR_64_02_2020/51.pdf
%(How CNN works) https://towardsdatascience.com/a-comprehensive-guide-to-convolutional-neural-networks-the-eli5-way-3bd2b1164a53
%(CNN on Mnist dataset) https://paperswithcode.com/paper/an-ensemble-of-simple-convolutional-neural
%https://paperswithcode.com/sota/image-classification-on-mnist
%https://medium.com/@muhammetbolat/supervised-unsupervised-techniques-on-mnist-dataset-3f2ffd4c41c5
%https://medium.com/swlh/introduction-to-computer-vision-with-mnist-2d31c6f4d9a6
%(TSNE on MNIST) https://colah.github.io/posts/2014-10-Visualizing-MNIST/
%https://altran-data-analytics.netlify.app/2018/07/03/mnist/


% @misc{Projectmodule,
%   organization = {Aalborg University},
%   url          = {https://moduler.aau.dk/course/2022-2023/DSNDATB521},
%   title        = {Theory-driven Data Analysis and Modeling},
%   urldate      = {2022-09-27}
% }
% @misc{MnistStatictics,
%   organization = {Paperswithcode},
%   url          = {https://paperswithcode.com/sota/image-classification-on-mnist},
%   title        = {Image Classification on MNIST},
%   urldate      = {2022-10-11}
% }
% @misc{WhatIsLogisticRegression,
%   organization = {IBM},
%   url          = {https://www.ibm.com/topics/logistic-regression},
%   title        = {What is logistic regression?},
%   urldate      = {2022-10-11}
% }
% @misc{SupervisedLearning,
%   organization = {IBM},
%   url          = {https://www.ibm.com/cloud/learn/supervised-learning},
%   title        = {Supervised learning},
%   urldate      = {2022-10-11}
% }
% @misc{LogisticRegressionProsAndCons,
%   organization = {Medium},
%   url          = {https://medium.datadriveninvestor.com/logistic-regression-essential-things-to-know-a4fe0bb8d10a},
%   title        = {Logistic Regression: Essential Things to Know},
%   urldate      = {2021-09-2}
% }
% @misc{K-NearestNeighborIBM,
%   organization = {IBM},
%   url          = {https://www.ibm.com/topics/knn#:~:text=The%20k%2Dnearest%20neighbors%20algorithm%2C%20also%20known%20as%20KNN%20or,of%20an%20individual%20data%20point},
%   title        = {K-Nearest Neighbors Algorithm},
%   urldate      = {2022-10-11}
% }
% @misc{SupportVectorMachines,
%   organization = {Towards Data Science},
%   url          = {https://towardsdatascience.com/support-vector-machine-introduction-to-machine-learning-algorithms-934a444fca47},
%   title        = {Support Vector Machine — Introduction to Machine Learning Algorithms},
%   urldate      = {2018-06-07}
% }
% @misc{SVMProsAndCons,
%   organization = {Roboticsbiz},
%   url          = {https://roboticsbiz.com/pros-and-cons-of-support-vector-machine-svm/},
%   title        = {Pros And Cons Of Support Vector Machine (SVM)},
%   urldate      = {2022-09-10}
% }
% @misc{CNNIBM,
%   organization = {IBM},
%   url          = {https://www.ibm.com/cloud/learn/convolutional-neural-networks},
%   title        = {How do convolutional neural networks work?},
%   urldate      = {2022-10-11}
% }
% @misc{CNNProsAndCons,
%   organization = {GeeksforGeeks},
%   url          = {https://www.geeksforgeeks.org/difference-between-ann-cnn-and-rnn/},
%   title        = {Difference between ANN, CNN and RNN},
%   urldate      = {2022-08-24}
% }
% @misc{,
%   organization = {},
%   url          = {},
%   title        = {},
%   urldate      = {}
% }
\chapter{Problem analysis}\label{cha:problem-analysis}
This chapter describes the motivation for the project, culminating in a problem statement.

\begin{itemize}
    \item Why Machine learning + Generalized pipeline overview?
    \item Why Feature Engineering: Dimensionality reduction
          \begin{itemize}
              \item Extraction
              \item Visualization
              \item Learning/debugging
          \end{itemize}
    \item Linear versus nonlinear methods
    \item Problem statement
\end{itemize}

\gls{ml} is a field of study within \gls{ai} concerned with learning from data. \gls{ml} is a complex and growing field used in many different areas. One of the reasons for the increasing interest in \gls{ml} is the increasing amount of available data. The amount of data collected in the world is increasing at an incredible rate, which is expected to continue in the future~\cite{data-never-sleeps}. Because \gls{ml} models are trained on data, the more data available, the better the models can be~\cite{Unreasonable-effectiveness-of-data-Norvig}.

However the complexity of \gls{ml} is also high. There are many different \gls{ml} models, each with different strengths and weaknesses. The choice of \gls{ml} model is therefore more often than not dependent on the data and the problem being solved. This means that choosing the right \gls{ml} model is a difficult task, which is further complicated by the fact that there is no single metric for evaluating the performance of a \gls{ml} model. The performance of a \gls{ml} model is often evaluated using multiple metrics, which makes it difficult to compare the performance of different models against each other.

It may be necessary to try out different \gls{ml} models to find the best one for a certain problem, which is a time-consuming process.

\section{Machine learning}\label{sec:machine-learning}
Figure~\ref{fig:basic-machine-learning-pipeline} illustrates a general \gls{ml} pipeline.

To compare and contrast linear and nonlinear methods through the lens of a model Efficiently the use of a pipeline is needed. A brief Introduction to pipelines, there use and how to introduce dimensionality reduction to them the purpose of this section.

A pipeline generally consists of four main steps: data, \gls{fe}, \gls{ml} model training, and model evaluation. First stem is the data step which essentially is the collection of data. The \gls{fe} step is used to transform the data into a form that is more suitable for the \gls{ml} model. The \gls{ml} model training step is used to train the \gls{ml} model on the data. The model evaluation step is used to evaluate the performance of the \gls{ml} model. The \gls{ml} model can then be used to make predictions on new data. \todo[inline]{Add text on the data step}


\begin{figure}[htb!]
    \centering
    \begin{tikzpicture}
        \node (b) [state] {feature engineering};
        \node (c) [state, shift={($(b.east)+(2cm,0)$)}] {model};
        \node (a) [state, shift={($(b.west)+(-2cm,0)$)}] {data};
        \node (d) [state, shift={($(c.east)+(2cm,0)$)}] {evaluation};
        \node (e) [state, shift={($(b.south)+(0,-2cm)$)}] {parameters};

        \draw[arrow, ->] (a) -- node[above,scale=.70,align=center,] {} (b);
        \draw[arrow, ->] (b) -- node[above,scale=.70,align=center,] {} (c);
        \draw[arrow, ->] (c) -- node[above,scale=.70,align=center,] {} (d);
        \draw[arrow, ->] (e) -- node[above,scale=.70,align=center,] {} (c);

        \draw[arrow, ->] (d.north) -- ++(0,0.75) -| (b);
        \draw[arrow, ->] (d.south) -- ++(0,-0.75) -| (c);
    \end{tikzpicture}
    \caption{Simplified machine learning pipeline}
    \label{fig:basic-machine-learning-pipeline}
\end{figure}

The \gls{fe} step is where Dimmensionality reduction will be relevant. This is where the dimmensions are reduced to increase perfomance of the model. This and the model is then evaluated to see if they can be tuned for better results. This tuning is generraly refered to as hyperparameter tuning. What the best hyper parameter tuning is depends on the data. This leads to the last decision before the problem statement, The choice of data.

\section{Data}\label{sec:data}
We have chosen to work with \gls{mnist}, which is descibed by its creators as
\textcquote{lecun-mnist-database}{The MNIST database of handwritten digits, available from this page, has a training set of 60,000 examples, and a test set of 10,000 examples. It is a subset of a larger set available from NIST. The digits have been size-normalized and centered in a fixed-size image}.


The reason why \gls{mnist} was chosen is because it has real-world uses inside of image recognition, particularly the assignment of recognizing handwritten numbers. The real-world use comes from detecting patterns in which numbers are written, which can be attributed due to the style of the writer, and / or artificial errors.


Another reason is because there are many projects, which revolve around recognizing handwritten digits using machine learning. That means that we can compare our results to other similar projects for better discussion of results. On a similar vein, the digits in the datset are \textcquote{lecun-mnist-database}{size-normalized and centered in a fixed-size image}, which means that Lecun et al.\ have worked with the data. The project could also have gone another route: collecting data, cleaning data, etc.\,but choosing the \gls{mnist} dataset has simplified the overall work for us, and allowing to focus more on the machine learning / dimensionality reduction part of the project. 


The group has also considered the Iris dataset, another introductory dataset, however that dataset might have too few data samples ~\cite{mnist-vs-iris}, which is also why the group has chosen to work with \gls{mnist}. As outlined before, there have been done some sort of preprocessing regarding the \gls{mnist}, which further solidified the choice regarding \gls{mnist}.


We have in this chapter presented the pipeline, and gave an overview of the different components of the pipeline. With this knowledege we can now move on to the problem statement.

%@misc{mnist-vs-iris,
% author       = {Matteo Kimura},
% url          = {https://lamfo-unb.github.io/2019/05/17/Introductory-Datasets/},
% title        = {Introductory Datasets},
% urldate      = {2022-11-15}
% }
% \section{Feature engineering}\label{sec:feature-engineering}
% % kort forklaring af FE, hvad er det 
% In machine learning \gls{fe} is used to transform the raw data into some type of data that is more suitable for the \gls{ml} model. A feature must derive from what type of data is given, and it is also tied to which model is being used. Some features are more appropriate for some types of models and vice versa. Feature engineering is the process of formulating the best fitted features given the task in hand, with the data and the chosen model~\cite{Feature-engineering-zheng}.

% There are many techniques for \gls{fe}, some examples on how to do \gls{fe} are: Imputation, handling outliers, scaling, dimensionality reduction. Imputation is the process of filling in missing values in the data. Most imputation is done by finding it by matrixes, by looking at other values in the dataset, a popular approach is k-nearest neighbors to find the missing values~\cite{imputation-for-tables-Biessmann}. Outliers are values that are far away from the rest of the data, and they can be a problem for some models, for both accuracy and inaccurate classification. It it therefore a good idea to eliminate the outliers, this is also a standard practice in most machine learning problems. There are many ways to handle outliers, one way is to remove them, another way is to replace them with the median or the mean of the data~\cite{outlier-perez}. 

% Scaling, also called feature normalization, is the process of transforming the data into a form that is more suitable for the \gls{ml} model. This is done by changing the range of the data, for example, if the data is in the range of 0-100, it can be scaled to be in the range of 0-1. This is done to make the data more suitable for the model, and to make it easier to compare the data. Scaling is also a standard practice in most machine learning problems. There are many ways to scale the data, one way is to use the min-max scaler, another way is to use variance scaling~\cite{Feature-engineering-zheng}.

% One of the goals of the dimensionality reduction methods is to counter the curse of dimensionality.
% According to Lee~\cite{nonlinear-dim-red-chapter-one}, the curse of dimensionality refers to "the number of data samples requried to estimate a function of several variables to a given accuracy on a given domain grows exponentially with the number of dimensions"~\cite{nonlinear-dim-red-chapter-one}. This means that machine learning models' performance might get affected by the huge amount of data that needs to be given.

% That is not optimal if we know that we can reduce the amount of data without losing too much information. On the contrary, the performance can perhaps be improved, partly because the size of the data gets reduced, and partly because the essence of the data is preserved. Dimensionality reduction will be further discussed in "chapter"~\ref{cha:theory}.


\section{Dimensionality reduction}\label{sec:dimensionality-reduction-problem}\todo[inline]{There are at least 2 ideas in this paragraph split it. Pointer.What is dimensionality reduction? What are its advantages? Computation? Interpretability? How does dimenstionality reduction work? This last question opens the gate to the section linear vs nonlinear}
Dimensionality is achieved by reducing the number of features(a feature is some measurable data) in a dataset. This can help to visualize the data or, within a large dataset, describe which data weighs heavier on the expected output of the \gls{ml} model. Time complexity is also a significant factor when discussing dimensionality reduction, as stated in "\textcquote{Analysis-of-Dimensionality-Reduction-Techniques-on-Big-Data}{Dimensionality reduction techniques can tremendously reduce the time complexity of training phase of ML algorithms hence reducing the burden of the machine learning algorithms.}. Time is a significant factor when discussing the \gls{ml} model, as the time complexity of the model is a significant factor in the overall time complexity of the system. Therefore, when reducing features in a dataset, it is crucial to find the features that are more relevant to the output of the model~\cite{Feature-engineering-zheng}. There are many ways to reduce the dimensionality of a dataset; the different types of dimensionality reduction will be further discussed in Chapter~\ref{cha:theory}.

There are many dimensionality reduction methods, and more are still being researched today~\cite{dimensionality-reduction-cheng}. Moreover, there are many different uses for these methods. An example of this could be to improve heuristic models for explaining the data from surveys better, through better visualizations as it helps to improve understanding~\cite{dimensionality-reduction-cheng}. Many more examples exist, but the main point is that dimensionality reduction is a handy tool in many different fields and a vital part of the \gls{ml} model. We will categorize dimensionality reduction into two general categories in the following section: Linear and nonlinear dimensionality reduction. 
% Forklar hvorfor vi har valgt at fokusere på dimensionalitets reduction 

\section{Linear versus nonlinear methods}\label{sec:linear-vs-nonlinear}

In this section \textbf{feature engeneering. Det er Sebastians sektion, som ikke er kommet ind endnu} we have seen some methods by which pre-processing can be achieved. In this section we will present a more specific family of feature engineering: dimensionality reduction. More specifically, in this project we will distinguish between linear and nonlinear methods. According to Lee, there are several distinctions that can be made for dimensionality reduction methods. We will not focus on them, but will choose to classify them as linear and nonlinear because it is "the straightest way to classify them" \cite{nonlinear-dim-red-chapter-two}.


The difference between linear and nonlinear methods is that linear methods express output as a linear combination of input data, and nonlinear methods do so in a nonlinear manner. This means that nonlinear methods are much more complex than the linear ones, but are also more computationally expensive \cite{nonlinear-dim-red-chapter-two}. The details regarding the difference between the methods will be presented in section \ref{sec:dimensionality-reduction}.

\subsection{Applications of dimensionality reduction methods}
The use of the linear and nonlinear dimensionality reduction methods can among others be seen in \cite{dimensionality-reduction-comparative-review} and \cite{tennenbaum}, where the methods have been tested on artificial and real-world datasets. As an example, it has been shown that artificial datasets such as the swiss-roll show that linear methods fail to find the intrinsic dimensionality of the data as opposed to nonlinear methods \cite{tennenbaum}. The research paper written by Jarkko shows that linear and nonlinear dimensionality reduction methods can be visualized on separate datasets \cite{dim-red-visual}, and visualization can provide an aid at analyzing which methods are better than others at finding an accurate lower representation of the data. The research paper \cite{dimensionality-reduction-comparative-review} compares the performance of linear and nonlinear dimensionality reduction methods with some machine learning model.


According to Laurens, there is a tendence that the real world data is nonlinear. This means that the linear methods are at disadvantage, because they are not able to capture the intrinsic dimensionality of the nonlinear data as good as nonlinear methods. However, he also states that nonlinear methods are not always able to outperform linear methods \cite{dimensionality-reduction-comparative-review}, which might be seen as counterintuitive, since nonlinear methods are supposed to outperform linear methods on nonlinear data.

\subsection{Relevance of linear vs nonlinear methods}
The reason why we focus on linear and nonlinear dimensionality reduction methods is because we want to explore whether these methods have a significant influence on the performance of a machine learning model. As outlined before, dimensionality reduction methods can be used to remove redundancy from data, which can improve the performance of a machine learning model. We have presented that nonlinear methods may work better on nonlinear datasets, and that they are suitable for nonlinear datasets, but according to Laurens \cite{dimensionality-reduction-comparative-review}, there might not be a major difference between the linear and nonlinear methods. That is is why we want to explore whether linear or nonlinear methods \textit{actually} have an impact on our machine learning model, based on the metrics that have been presented in this chapter. 


In this section we have presented a short overview of the dimensionality reduction methods, herunder linear and nonlinear methods. We have also presented some applications of the methods, and provided a reason for why we want to explore the methods in the project. \textbf{Det kan tages med eller slettes}: In this section we have also presented a potential challenge that we might face in the project: the fact that nonlinear methods may not be better than linear methods.

% @misc{dim-red-visual,
% title={{Dimensionality reduction for visual exploration of similarity structures}},
% author={Venna, Jarkko},
% year={2007},
% language={English},
% pages={81, [115]},
% publisher={Helsinki University of Technology},
% type={Doctoral thesis},
% keywords={dimensionality reduction, exploratory data analysis, information retrieval, information visualization, manifold learning, Markov Chain Monte Carlo},
% isbn={978-951-22-8752-9},
% series={Dissertations in computer and information science. Report D; 20},
% issn={1459-7020},
% url={}
% }

%@book{nonlinear-dim-red-chapter-one,
% author="Lee, John A.
% and Verleysen, Michel",
% editor="Lee, John A.
% and Verleysen, Michel",
% title="High-Dimensional Data",
% bookTitle="Nonlinear Dimensionality Reduction",
% year="2007",
% publisher="Springer New York",
% address="New York, NY",
% pages="1--16",
% isbn="978-0-387-39351-3",
% doi="10.1007/978-0-387-39351-3_1",
% url="https://doi.org/10.1007/978-0-387-39351-3_1"
% }

% @book{nonlinear-dim-red-chapter-two,
% author="Lee, John A.
% and Verleysen, Michel",
% editor="Lee, John A.
% and Verleysen, Michel",
% title="Characteristics of an Analysis Method",
% bookTitle="Nonlinear Dimensionality Reduction",
% year="2007",
% publisher="Springer New York",
% address="New York, NY",
% pages="17--45",
% isbn="978-0-387-39351-3",
% doi="10.1007/978-0-387-39351-3_2",
% url="https://doi.org/10.1007/978-0-387-39351-3_2"
% }

% @book{nonlinear-dim-red-chapter-three,
% author="Lee, John A.
% and Verleysen, Michel",
% editor="Lee, John A.
% and Verleysen, Michel",
% title="Estimation of the Intrinsic Dimension",
% bookTitle="Nonlinear Dimensionality Reduction",
% year="2007",
% publisher="Springer New York",
% address="New York, NY",
% pages="47--67",
% isbn="978-0-387-39351-3",
% doi="10.1007/978-0-387-39351-3_3",
% url="https://doi.org/10.1007/978-0-387-39351-3_3"
% }



\section{Problem Statement}\label{sec:problem-statement}

\newacronym{cnn}{CNN}{Convolutional Neural Network}
\newacronym{mnist}{MNIST}{Modified National Institute of Standards and Technology}
\newacronym{nn}{NN}{Neural Network}
\newacronym{ml}{ML}{Machine Learning}

This project explores the impact of data preproccesing on the performance of machine learning using a logistic regression model versus \gls{cnn} for the computer vision problem of image classification and recognition. The data preprocessing is done through dimensionality reduction on augmented data from the \gls{mnist} database, and the machine learning models are trained on the reduced data. The performance metrics used to evaluate the models are accuracy, precision, recall, and F1 score\question{Or something else? Placeholder metrics}. and of course explainability and speed/size of the models.

\subsection{Tools}\label{subsec:tools}
Data preproccessing, data augmentation and feature engineering

Use Keras to build a \gls{ml} model.

Computer vision

Explainability - \gls{nn} vs other \gls{ml} algorithms

Humans vs computers in \gls{nn}. Why are humans good with little training, and computers only accceptable with much more training? Consider perhaps domains (recongnizing epsilon vs. recognizing a 3)

\supervisor{How do we determine recall and precision for logistic regression?}
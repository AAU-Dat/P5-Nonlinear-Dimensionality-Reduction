\subsubsection{Reshape}\label{subsec:preprocessing-reshape}
The data, when loaded in at first, is shaped as a long array of 47.040.000 integers that represent the values of the pixels, and then another array of 60.000 integers that represents what number a given image is. The images are considered a 28x28 matrix, meaning that the first 784 integers represent the first image, and the successive 784 integers represent the second image. Working with the data in this format proved difficult and confusing, as tracking what image was being worked on was hard. Because of this, it was decided to reshape the data to make it easier to understand. The data was reshaped to a two-dimensional array, where the first dimension represented images, and the second dimension represented the array of pixels and their values. From this, there was a clear distinction between each image.

  
An example, to further explain the decision for reshaping the data, can be seen in~\cite{scikit-learn-PCA}, where it is stated that the function \texttt{fit} expects: \textcquote{scikit-learn-PCA}{X: array-like of shape (n\_samples, n\_features)}, which means that data is expected to be sectioned into an array of samples(images), and an array of features. When the data was not reshaped, it was impossible to use the \texttt{fit} function, as the data was not in the correct format. The data would have n\_samples, that was of the size 47.040.000, which is not the actual number of samples (images), but rather the number of pixels in all 60.000 images.
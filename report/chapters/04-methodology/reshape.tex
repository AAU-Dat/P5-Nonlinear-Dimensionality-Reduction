\subsubsection{Reshape}\label{subsec:preprocessing-reshape}
The data, when loaded in at first, is shaped as a long array of 47.040.000 integers representing the values of the pixels in all 60.000 images, and then another array of 60.000 integers representing what number a given image is. The images are in a 28x28 matrix, meaning that the first 784 integers represent the first image, and the successive 784 integers represent the second image. This format was not compatible with \gls{sckit-learn}, where it is stated that the function \texttt{fit} expects: \textcquote{scikit-learn-PCA}{X: array-like of shape (n\_samples, n\_features)}, which means that the function expects the data to be in array of samples(images), and an array of features(pixels). Because of the input \texttt{fit} expects, the data needed to be reshaped into a 60.000x784 matrix, where each row represents one image, and each column represents a pixel. The implementation uses the \texttt{reshape} function in \texttt{numpy}. From this, there is a clear distinction between each image.
  
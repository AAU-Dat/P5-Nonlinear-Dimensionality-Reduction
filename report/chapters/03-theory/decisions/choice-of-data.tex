\section{Data}\label{sec:data}
We have chosen to work with \gls{mnist}, which its creators describe as
\textcquote{lecun-mnist-database}{The MNIST database of handwritten digits, available from this page, has a training set of 60,000 examples, and a test set of 10,000 examples. It is a subset of a larger set available from NIST. The digits have been size-normalized and centered in a fixed-size image}.


\gls{mnist} was chosen because it has real-world uses inside of image recognition, particularly recognizing handwritten numbers. The real-world use comes from detecting patterns which numbers are written, which can be attributed due to the style of the writer and/or artificial errors.


Another reason is that many projects revolve around recognizing handwritten digits using machine learning. That means we can compare our results to similar projects for better discussion of results. Similarly, the digits in the dataset are \textcquote{lecun-mnist-database}{size-normalized and centered in a fixed-size image}, which means that Lecun et al.\ has worked with the data. The project could also have gone another route, such as collecting and cleaning data but choosing the \gls{mnist} dataset has simplified the overall work for us and allowed us to focus more on the machine learning/dimensionality reduction part of the project. 


The group has also considered the Iris dataset, another introductory dataset. However that dataset might have too few data samples ~\cite{mnist-vs-iris}, which is also why the group has chosen to work with \gls{mnist}. As outlined before, there has been some preprocessing regarding the \gls{mnist}, which further solidified the choice regarding \gls{mnist} \todo[inline]{Do not look only at the Iris example. Think of other datasets. This section could be moved to methodology and explained in more detail}.
\section{Data}\label{sec:data}
In this project, it was chosen to work with \gls{mnist}. \gls{mnist} was chosen because it has real-world uses inside image recognition, particularly recognizing handwritten numbers. The real-world use comes from detecting patterns in how numbers are written, which can be attributed to the writer's style, artificial errors, or both. 

Another reason is that many projects revolve around recognizing handwritten digits using machine learning, therefore is well documented, which makes it easy to find information about it. That means this project can compare the results to similar projects for better discussion of results. The dataset is also tiny, which means it is easy to work with and does not require a lot of computing power~\cite{lecun-mnist-database}.

The project could also have gone another route, such as collecting and cleaning data. However, choosing the \gls{mnist} dataset has simplified the overall work and allowed more focus on the project's machine learning/dimensionality reduction part. 

The group has also considered the Iris dataset and the CIFAR-10 dataset. However, the Iris dataset might have too few data samples~\cite{mnist-vs-iris}, and the CIFAR-10 dataset would be too complex for this project~\cite{datasets-uniqtech}. Therefore the group has chosen to work with \gls{mnist}. As outlined before, there has been some preprocessing regarding the \gls{mnist}, which further solidified the choice of \gls{mnist}.




%which its creators describe as \textcquote{lecun-mnist-database}{The MNIST database of handwritten digits, available from this page, has a training set of 60,000 examples, and a test set of 10,000 examples. It is a subset of a larger set available from NIST. The digits have been size-normalized and centered in a fixed-size image}.

%Similarly, the digits in the dataset are \textcquote{lecun-mnist-database}{size-normalized and centered in a fixed-size image}, which means that Lecun et al.\ has worked with the data. 
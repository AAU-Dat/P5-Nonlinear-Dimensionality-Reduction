\section{Images}\label{sec:images}
In this project, the \gls{mnist} dataset has been chosen because it is commonly used in image recognition tasks, specifically for recognizing handwritten digits. This dataset is well-documented, which allows for comparing the results to similar projects. It is small in size, making it easy to work with and not requiring a lot of computational power. \gls{mnist} has 60000 training samples and 10000 test samples~\cite{lecun-mnist-database}.

It was considered to use the Iris and \gls{cifar} datasets, but the Iris dataset may have too few data samples~\cite{mnist-vs-iris}, and the \gls{cifar} dataset could be too complex for this project~\cite{fashion-mnist}. The \gls{mnist} dataset consists of images, each composed of pixels. In a grayscale image, each pixel has a value between 0 and 255, with 0 representing white and 255 representing black~\cite{lecun-mnist-database}.

This project aims to demonstrate dimensionality reduction by representing each pixel as a feature in the image. As the size of the images increases, this quickly becomes a large number of features.
\subsubsection{Data augmentation}\label{subsec:data-augmentation}
This section will describe data augmentation, its relevance, and which specific augmentation will be used.

\textcquote{ShortenConnor2019AsoI}{Data augmentation in data analysis are techniques used to increase the amount of data by adding slightly modified copies of already existing data or newly created synthetic data from existing data. It acts as a regularizer and helps reduce overfitting when training a machine learning model.}

Data augmentation has many reasons, typically to make a model more general and less overfitted. This happens because we only have so much data, often from multiple sources in the real world. So augmentation is to make a model less reliant on these less-than-ideal data sources~\cite {MAHARANA202291}.
In the case of this project, augmentation is chosen, so it's possible to show how the \gls{fe} and the model handle different data shapes while still having a baseline. This helps us make more general inferences and intuition about \gls{fe} and \gls{ml}. It will, in particular, show which \gls{fe} methods best handle different data shapes. It will also help show how general augmentation can affect performance \gls{ml} models and \gls{fe}.
We chose one type of augmentation from the following list of augmentations. Rotation, removing pixels, scaling. These were selected because they are all well-suited for images since they create variations of images~\cite{gonzalez2008digital}. There exist others, but they were not considered for this project. 

Rotation and scaling augmentation was chosen in this project because it believed they would create the most exciting challenges for the model and \gls{fe}. 

All the chosen preprocesses for the project are now described. The following section concerns the next point in \gls{fe} dimensionality reduction. 




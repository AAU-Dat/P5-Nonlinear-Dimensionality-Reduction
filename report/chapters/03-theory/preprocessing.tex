\section{Preprocessing}\label{sec:preprocessing}
In this section, the theory of preprocessing and what effects it has on the \gls{ml} model is discussed.


%What is preproccessing?
%What effects does it have on the model/data?
%Why do we need it?
%What are the steps?
%What are the tools?
%What are the results?

\subsection{Definition of preprocessing}\label{subsec:preprocessing-definition}
The proper definition of preprocessing or data preparation varies depending on the source. The explanation that will be used in this report is given as follows:

\textcquote{data-preparation-for-data-mining}{Data preparation comprises those techniques concerned with analyzing raw data so as to yield quality data, mainly including data collecting, data integration, data transformation, data cleaning, data reduction, and data discretization.}

From this, it can be gathered that preprocessing is a broad term that can be divided into several subcategories. Not all subcategories will be relevant to this project; therefore, the following sections will only discuss appropriate subcategories of preprocessing.
  

\subsection{Reasons for preprocessing}\label{subsec:preprocessing-reasons}
As stated in the definition, preprocessing is used, among other reasons, to yield quality data. The importance of this stems from the fact that real-world data is only sometimes clean or complete. Meaning that can be a lot of noise, which is data containing errors or outliers. This noise can be removed by preprocessing, which creates a more accurate, higher quality, and smaller dataset to gather information. This results in a reduced amount of data that the model is trained on, but the data it is trained on should be more accurate, which should then train the model more accurately and efficiently~\cite{data-preparation-for-data-mining}.


Data collected may be in a form or shape that is not compatible with the process that is needed to work with it. Therefore, to use the data, it may need to be transformed into a form that fits with the process. Transformation of data can be many things, as~\cite{Data-preprocessing-for-flight-delays} mentions normalization as a part of the transformation, to scale the data so that it fits into a new range, Subsection \ref{sec:normalization} will give a more detailed explanation of normalization.

  
  
\subsection{Steps of preprocessing}\label{subsec:preprocessing-steps}
The amount of preprocessing depends on what is needed and what is available. In the report~\cite{Data-preprocessing-for-flight-delays}, these steps are shown. 

They start by cleaning data, transforming it, reducing it, and balancing it. This section will explain the theory of the steps used in this report.

\section{Normalization}\label{sec:normalization}
Scaling, also called feature normalization, is the process of transforming the data into a form that is more suitable for the \gls{ml} model. This is done by changing the range of the data, for example, if the data is in the range of 0-100, it can be scaled to be in the range of 0-1. This is done to make the data more suitable for the model, and to make it easier to compare the data. Scaling is also a standard practice in most machine learning problems. There are many ways to scale the data, one way is to use the min-max scaler, another way is to use variance scaling~\cite{Feature-engineering-zheng}.
\subsection{Min-max scaler}
The min-max scaler is a simple way to scale the data. It scales the data to be in the range of 0-1. The formula for the min-max scaler is:
\begin{equation}
    x_{scaled} = \frac{x - x_{min}}{x_{max} - x_{min}}
\end{equation}
where $x$ is the original value, $x_{scaled}$ is the scaled value, $x_{min}$ is the minimum value in the data, and $x_{max}$ is the maximum value in the data. The min-max scaler is a simple way to scale the data, but it is not robust to outliers. If there are outliers in the data, the min-max scaler will scale the data to be in the range of 0-1, but the outliers will be scaled to be very close to 0 or 1. This can be a problem if the outliers are important to the model. The min-max scaler is also sensitive to the presence of zeros in the data. If there are zeros in the data, the min-max scaler will scale the data to be in the range of 0-1, but the zeros will be scaled to be 0. This can be a problem if the zeros are important to the model. 

\subsubsection{Data augmentation}\label{subsec:data-augmentation}
This section will describe data augmentation, its relevance, and which specific augmentation will be used.

\textcquote{ShortenConnor2019AsoI}{Data augmentation in data analysis are techniques used to increase the amount of data by adding slightly modified copies of already existing data or newly created synthetic data from existing data. It acts as a regularizer and helps reduce overfitting when training a machine learning model.}

Data augmentation has many reasons, typically to make a model more general and less overfitted. This happens because we only have so much data, often from multiple sources in the real world. So augmentation is to make a model less reliant on these less-than-ideal data sources~\cite {MAHARANA202291}.
In the case of this project, augmentation is chosen, so it's possible to show how the \gls{fe} and the model handle different data shapes while still having a baseline. This helps us make more general inferences and intuition about \gls{fe} and \gls{ml}. It will, in particular, show which \gls{fe} methods best handle different data shapes. It will also help show how general augmentation can affect performance \gls{ml} models and \gls{fe}.
We chose one type of augmentation from the following list of augmentations. Rotation, removing pixels, scaling. These were selected because they are all well-suited for images since they create variations of images~\cite{gonzalez2008digital}. There exist others, but they were not considered for this project. 

Rotation and scaling augmentation was chosen in this project because it believed they would create the most exciting challenges for the model and \gls{fe}. 

All the chosen preprocesses for the project are now described. The following section concerns the next point in \gls{fe} dimensionality reduction. 

% @article{ShortenConnor2019AsoI,
% title = {A survey on Image Data Augmentation for Deep Learning},
% year = {2019},
% author = {Shorten, Connor and Khoshgoftaar, Taghi M.},
% }

% @article{MAHARANA202291,
% title = {A review: Data pre-processing and data augmentation techniques},
% journal = {Global Transitions Proceedings},
% year = {2022},
% doi = {https://doi.org/10.1016/j.gltp.2022.04.020},
% url = {https://www.sciencedirect.com/science/article/pii/S2666285X22000565},
% author = {Kiran Maharana and Surajit Mondal and Bhushankumar Nemade},
% }

% @book{gonzalez2008digital,
%   author = {Gonzalez, Rafael C. and Woods, Richard E.},
%   isbn = {9780131687288 013168728X 9780135052679 013505267X},
%   title = {Digital image processing},
%   year = {2008}
% }





%Explain why this is needed, for other reasons than just making it easier to understand.
%Does scikit methods need it to be in this format?
%Easier to use a function on a single image  

%Reshape
%normalize/Scale - StandardScaler
    %Look at others
  

%focus on cleaning or dim reduction
%Is augmentation a part of preproccessing?
%What is the difference between cleaning and dim reduction?
%Transforming data to fit our current needs
  

%Data preparation comprises those techniques concerned with analyzing raw data so as to yield quality data, mainly including data collecting, data integration, data transformation, data cleaning, data reduction, and data discretization.
    %We want data preprocessing, due to real world data being incomplete, noisy, and inconsistent.
    %Generates a smaller dataset to work with, which makes it more efficient for datamining.

%https://www.frontiersin.org/articles/10.3389/fbioe.2020.00260/full
    %Data preprocessing subcategories: (1) Ground reaction force (GRF) filtering, (2) time derivative, (3) time normalization, (4) data reduction, (5) weight normalization, and (6) data scaling.
 

%https://praveenkds.medium.com/data-preparation-for-machine-learning-data-cleaning-data-transformation-data-reduction-c4c86c4471a1
    %Data cleaning: removing noise, outliers, and missing values.
    %Data transformation: scaling, normalization, and discretization.
    %Data reduction: feature selection and feature extraction.
\subsubsection{Normalization}\label{sec:normalization}
Normalization, also called scaling, is the process of scaling the data. This is done by changing the range of the data; for example, if the data is in the field of 0-100, it can be scaled to be in the range of 0-1. This is done to make the data more suitable for the model and to make it easier to compare the data. Scaling is also a standard practice in most \gls{ml} problems. There are many ways to scale the data; for example, there is a min-max scaler~\cite{Feature-engineering-zheng}.


\paragraph{Min-max scaler}\label{subsec:min-max}
The min-max scaler is a method that transforms the data to be in the range between 0-1 by subtracting the minimum value and dividing by the scope of the data. The formula for this is:

\begin{equation}
    x_{scaled} = \frac{x - x_{min}}{x_{max} - x_{min}}
\end{equation}

Where $x$ is the original value, $x_{scaled}$ is the scaled value, $x_{min}$ is the minimum value in the data, in the case of this project, this will be 0, and $x_{max}$ is the maximum value in the data, in this project's case, this would be 255.

The min-max scaler is a simple way to scale the data, but it is not robust to outliers. If there are outliers in the data, the min-max scaler will scale the data to be in the range of 0-1, but the outliers will be scaled very close to 0 or 1. While other data will be clustered together, this can be a problem if the outliers are essential to the model. The min-max scaler is also sensitive to the presence of zeros in the data. If there are zeros in the data, the zeros will remain zero, and this can be a problem if the zeros are essential to the model, however for this project, this will not be a concern, as this does not affect the model~\cite{Feature-engineering-zheng}.

\subparagraph{Variance scaling}\label{sec:variance-scaling}
Variance scaling is a more robust way to scale the data, but it is more complex. The variance scaling is a method that transforms the data to have a mean of 0 and a variance of 1, by subtracting the mean and dividing by the standard deviation. The formula for the variance scaling is:

\begin{equation}
    x_{scaled} = \frac{x - mean(x)}{\sqrt{var(x)}}
\end{equation}

Where $x$ is the original value, $x_{scaled}$ is the scaled value, $mean(x)$ is the mean of the data, and $var(x)$ is the variance of the data~\cite{Feature-engineering-zheng}.
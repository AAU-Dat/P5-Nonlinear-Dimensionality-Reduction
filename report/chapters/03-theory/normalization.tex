\section{Normalization}\label{sec:normalization}
The feature normalization, also called scaling, is the process of transforming the data into a form that is more suitable for the \gls{ml} model. This is done by changing the range of the data, for example, if the data is in the range of 0-100, it can be scaled to be in the range of 0-1. This is done to make the data more suitable for the model, and to make it easier to compare the data. Scaling is also a standard practice in most machine learning problems. There are many ways to scale the data, one way is to use the min-max scaler, another way is to use variance scaling~\cite{Feature-engineering-zheng}. 
\subsubsection{Min-max scaler}
The min-max scaler is a method that transforms the data to be in the range of 0-1, by subtracting the minimum value and dividing by the range of the data. The variance scaling is a method that transforms the data to have a mean of 0 and a variance of 1, by subtracting the mean and dividing by the standard deviation.

The formula for the min-max scaler is:
\begin{equation}
    x_{scaled} = \frac{x - x_{min}}{x_{max} - x_{min}}
\end{equation}

Where $x$ is the original value, $x_{scaled}$ is the scaled value, $x_{min}$ is the minimum value in the data, in this projects case this will be 0, and $x_{max}$ is the maximum value in the data, in this projects case this would be 255. 

The min-max scaler is a simple way to scale the data, but it is not robust to outliers. If there are outliers in the data, the min-max scaler will scale the data to be in the range of 0-1, but the outliers will be scaled to be very close to 0 or 1. This can be a problem if the outliers are important to the model. The min-max scaler is also sensitive to the presence of zeros in the data. If there are zeros in the data, the min-max scaler will scale the data to be in the range of 0-1, but the zeros will be scaled to be 0. This can be a problem if the zeros are important to the model, however in this project this will not be a concern, as this does not effect the model. 
\subsubsection{Variance scaling}
Variance scaling is a more robust way to scale the data, but it is more complex. The variance scaling is a method that transforms the data to have a mean of 0 and a variance of 1, by subtracting the mean and dividing by the standard deviation. The formula for the variance scaling is:
\begin{equation}
    x_{scaled} = \frac{x - mean(x)}{\sqrt{var(x)}}
\end{equation}
Where $x$ is the original value, $x_{scaled}$ is the scaled value, $mean(x)$ is the mean of the data, and $var(x)$ is the variance of the data. 

Our data is in the range of 0-255, so we will use the min-max scaler to scale the data to be in the range of 0-1. The min-max scaler is a simple way to scale the data, and it is not sensitive to the presence of zeros in the data. The min-max scaler is also not sensitive to outliers, so it will not be a problem if there are outliers in the data. 
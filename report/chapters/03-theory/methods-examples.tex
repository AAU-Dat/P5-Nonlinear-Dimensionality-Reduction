\section{Examples of methods}
In this section we will present how the linear and nonlinear methods behave on linear and nonlinear data.




\subsection{Linear data}
As linear data we will use the Iris dataset, which contains three different species of Iris flowers, with 50 examples each. The dataset has four dimensions, which are the length and width of the sepals, and petals ~\cite{iris-dataset}. The dataset is a simple and well-known dataset, which has been used in many machine learning papers ~\cite{iris-dataset}, and should give a fair comparison between the methods.[not sure about this]


On figure \ref{I don't have a ref} the iris flowers are represented on a 3d plot. From the figure one can see that the species setosa is linearly separable from the other two species, as opposed to them.





\subsection{Nonlinear data}
As nonlinear data we will construct two different classes of cirlces, an inner- and outer circle.

% @misc{iris-dataset,
% author = "Dua, Dheeru and Graff, Casey",
% year = "2017",
% title = "{UCI} Machine Learning Repository",
% url = "http://archive.ics.uci.edu/ml",
% institution = "University of California, Irvine, School of Information and Computer Sciences" }
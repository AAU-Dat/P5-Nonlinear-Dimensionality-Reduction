\section{Cross-validation}
\todo[inline, color=gray]{The sources are commented out in the file}
\todo{I do not know if I need an introduction}
In order to determine which parameters a learning model should use, so as to not overfit, one can make use of cross-validation. Cross-validation implies that the training data should be split into three sets: training data, validation data(comprising of training data), and test data. By partitioning the training data into two sets, the model can learn on the training data, and evaluate on the validation data. The model can be tuned with the validation data, and, after the model is optimized, evaluated on the test data. Such approach is useful, because the model gets evaluated based on data it has never seen \cite{scikit-learn}.


\todo[inline, color=red]{I do not know which one of the bibliographies to use for sklearn}
%This is written as @inproceedings in the original bib
% @article{sklearn\_api,
%   author    = {Lars Buitinck and Gilles Louppe and Mathieu Blondel and
%                Fabian Pedregosa and Andreas Mueller and Olivier Grisel and
%                Vlad Niculae and Peter Prettenhofer and Alexandre Gramfort
%                and Jaques Grobler and Robert Layton and Jake VanderPlas and
%                Arnaud Joly and Brian Holt and Ga{\"{e}}l Varoquaux},
%   title     = {{API} design for machine learning software: experiences from the scikit-learn
%                project},
%   booktitle = {ECML PKDD Workshop: Languages for Data Mining and Machine Learning},
%   year      = {2013},
%   pages = {108--122},
% }    

% @article{scikit-learn,
%     title={Scikit-learn: Machine Learning in {P}ython},
%     author={Pedregosa, F. and Varoquaux, G. and Gramfort, A. and Michel, V.
%     and Thirion, B. and Grisel, O. and Blondel, M. and Prettenhofer, P.
%     and Weiss, R. and Dubourg, V. and Vanderplas, J. and Passos, A. and
%     Cournapeau, D. and Brucher, M. and Perrot, M. and Duchesnay, E.},
%     journal={Journal of Machine Learning Research},
%     volume={12},
%     pages={2825--2830},
%     year={2011}
%     }


%It is written as @inBook in the original bib
% @book{Refaeilzadeh2009,
% author="Refaeilzadeh, Payam
% and Tang, Lei
% and Liu, Huan",
% editor="LIU, LING
% and {\"O}ZSU, M. TAMER",
% title="Cross-Validation",
% bookTitle="Encyclopedia of Database Systems",
% year="2009",
% publisher="Springer US",
% address="Boston, MA",
% pages="532--538",
% isbn="978-0-387-39940-9",
% doi="10.1007/978-0-387-39940-9_565",
% url="https://doi.org/10.1007/978-0-387-39940-9_565"
% }


%How to handle classifications (Do we treat each class as a separate part of the dataset?)
The basic form of cross-validation is the basic form of K-fold cross-validation \cite{Refaeilzadeh2009}. The method splits the data into $k$ equal segments/folds. For each fold, the learning model is then trained on $k-1$ folds and tested on the $k$'th fold. The final score is the average of the $k$'th scores. The metrics used to evaluate the model can be accuracy, precision, recall, and/or F1-score, as presented in the report. The advantage of K-fold is that it can compute optimal hyperparameters for the model by trying the different configurations that are given, but the disadvantage is that the method can be computationally expensive \cite{scikit-learn}, especially for the MNSIT dataset.

\todo[color=blue!30]{On sklearn(about cross-validation) there is a nice figure which shows K-fold. We could use that}

It is worth noting that K-fold alone does not stratify the data. Stratifying, or shuffling, the data may be necessary so as each fold is a good representation of the dataset, more specifically, each fold should have roughly the same distribution of samples of each class. There is a risk where the folds contain imbalanced amounts of samples for each data, which may affect the overall model performance, considering the size of the MNIST dataset.



%Why
Cross-validation can be effectively used to compare the performance of more learning models, or of the different parameters of a single model. In the project we will mostly use cross-validation for tuning the SVM model. The choice of $k$ should be made through trial and error. However, the train and data samples need to be large enough so as to be statistically representative of the the data set. A common value for $k$ is 5 or 10 \cite{introduction-to-statistical-learning}. Another paper suggests that strattified 10-fold cross-validation is a good \cite{Refaeilzadeh2009} \cite{kohavi1995study}.
\todo[inline,color=gray]{We could also use cross-validation wrt. dim. red in order to find the best dim. red. according to metrics. Then we can train the model with the best dim. red. methods, and perhaps have a hypothetical best model with best dim. red.}


%This has @inproceedings in the original
% @book{kohavi1995study,
%   title={A study of cross-validation and bootstrap for accuracy estimation and model selection},
%   author={Kohavi, Ron and others},
%   booktitle={Ijcai},
%   volume={14},
%   number={2},
%   pages={1137--1145},
%   year={1995},
%   organization={Montreal, Canada}
% }


% @book{introduction-to-statistical-learning,
%     title={An Introduction to Statistical Learning: with Applications in R},
%     author={James, Gareth and Witten, Daniela and Hastie, Trevor and Tibshirani, Robert},
%     year={2013},
%     publisher={Springer}
%     isbn = {9871461471370}
%     month = {08}
% }

\chapter{Examples}\label{app:examples}
When working in \LaTeX\ we have basic text, such as this, and non-basic elements called 'floats'. They are called floats because they float about the page, trying to be as unobtrusive as possible. The common floats used by us Computer Science students are: Figures, Tables, Listings, and potentially Algorithms.

\section{Figures}
Figures include an image or a graphical frame of some sort.


\begin{figure}[htb!]
    \centering
    \includegraphics[width=0.7\textwidth]{example-image-a}
    \caption{An example of a figure float with width at 70\% text width - \textbf{do not} use scale.}
    \label{fig:example-image-1}
\end{figure}


\clearpage
\section{Tables}
Tables are often more difficult than figures, but they can look gorgeous.


\begin{table}[htb!]
    \centering
    \begin{tabular}{lcc>{\bfseries}c}
        \toprule
        Features             & Events & Threads & Protothreads \\
        \midrule
        Control structures   & no     & yes     & yes          \\
        Debug stack retained & no     & yes     & yes          \\
        Implicit locking     & yes    & no      & yes          \\
        Preemption           & no     & yes     & no           \\
        Automatic variables  & no     & yes     & no           \\
        \bottomrule
    \end{tabular}
    \caption{An example table. Do not use \textit{vertical} (|) lines if you can avoid it.}
    \label{tab:example-table-1}
\end{table}


\section{Listings/Algorithms}
We can use the listing float for both code and pseudocode, but in case you want to distinguish between them, the Algorithm environment is a good substitute for pseudocode.

\begin{listing}[htb!]
    \begin{minted}[autogobble]{c}
        #include <stdio.h>

        int main() {
            printf("Hello World!");
            return 0;
        }
    \end{minted}
    \caption{Example of C code with standard styling.}
    \label{lst:example-listing-1}
\end{listing}

\begin{listing}[htb!]
    \begin{minted}[highlightlines={1,7}]{python}
        import numpy as np

        def incmatrix(genl1,genl2):
            m = len(genl1)
            n = len(genl2)
            M = None #to become the incidence matrix
            VT = np.zeros((n*m,1), int)  #dummy variable
    \end{minted}
    \caption{Example of python code with custom styling from preamble and lines highlight.}
    \label{lst:example-listing-2}
\end{listing}

For pseudocode, if you don't want to use the listing float, you can use the algorithm float instead.

\begin{algorithm}
    \begin{codebox}
        \Procname{$\proc{Insertion-Sort}(A)$}
        \li \For $j \gets 2$ \To $\attrib{A}{length}$
        \label{li:ins-sort-for}
        \li \Do
        $\id{key} \gets A[j]$ \label{li:ins-sort-pick}
        \label{li:ins-sort-for-body-begin}
        \li \Comment Insert $A[j]$ into the sorted sequence
        $A[1 \twodots j-1]$.
        \li $i \gets j-1$ \label{li:ins-sort-find-begin}
        \li \While $i > 0$ and $A[i] > \id{key}$
        \label{li:ins-sort-while}
        \li \Do
        $A[i+1] \gets A[i]$ \label{li:ins-sort-while-begin}
        \li $i \gets i-1$ \label{li:ins-sort-find-end}
        \label{li:ins-sort-while-end}
        \End
        \li $A[i+1] \gets \id{key}$ \label{li:ins-sort-ins}
        \label{li:ins-sort-for-body-end}
        \End
    \end{codebox}
    \caption{Example of pseudocode (codebox) in an algorithm float.}
    \label{alg:example-algorithm-1}
\end{algorithm}


\section{Floats and content}
Note that Figures use includegraphics, Tables use tabular, Listings use minted and Algorithms use codebox in the above example. This is NOT a hard and fast rule. You could have an Algorithm with an includegraphics screengrab, or a Listing with a tabular to create columns for code side by side comparisons.
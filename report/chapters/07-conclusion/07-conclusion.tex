\chapter{Conclusion}\label{cha:conclusion}
Based on the discussion in \autoref{cha:discussion}, the results from \autoref{cha:results} and the problem statement in \autoref{sec:problem-statement}, we conclude:

The project provided valuable insights into data analysis and \gls{ml}. We learned how to construct and develop a comprehensive \gls{ml} pipeline and gained a deeper understanding of various dimensionality reduction techniques and their applications. We also gained experience with the \gls{svm} classification algorithm and learned how to use it to make predictions and evaluate performance.

We found that the linear methods, \gls{pca} and \gls{lda}, performed particularly well in terms of speed on the \gls{mnist} dataset when used with an \gls{svm} classifier, with an acceptable loss of accuracy for \gls{pca} in particular, while \gls{lda} has an advantage at low dimensions.

Nonlinear methods, such as \gls{isomap} and \gls{kpca}, did not outperform linear methods in this case but may have the potential to be useful in certain situations.

Overall, this project has given us a strong foundation for further study and research in data analysis and provided a good introduction to working theoretically in a scientific setting with \gls{ml}.

% This chapter contains the concluding remarks of the project. It is based on the discussion in chapter \ref{cha:discussion}, the results from chapter \ref{cha:results} and the problem statement in section \ref{sec:problem-statement}. The chapter concludes with a reflection and perspectives for future work.
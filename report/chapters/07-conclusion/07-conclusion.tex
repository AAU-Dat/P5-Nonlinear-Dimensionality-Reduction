\chapter{Conclusion}\label{cha:conclusion}
\chapter{Future works}\label{sec:future-works}
With the insights gained from the project, we have determined several points of interest worth looking into in the future.

Initially, the project included plans for data augmentation in the pipeline, such as slight rotation of the images or artificial errors by whitening pixels of the numbers, making this a prime candidate for future work. These augmentations would artificially increase the size of the dataset, allowing for more splits, more model accuracy, and possibly making the data more nonlinear.

The \gls{mnist} dataset is relatively simple, so it may be interesting to explore more complex datasets and study the impact of dimensionality reduction on those. Some potential datasets already mentioned in the report are the \gls{cifar} and \gls{fashion-mnist} datasets, which may increase the potential of the nonlinear approaches. Alternatively, other data types, for example, audio, would be interesting to look at because it is very different from \gls{mnist}.

Different dimensionality reduction methods would also be interesting. Especially methods not covered in depth for this report, such as \gls{nmf}, would be exciting.

Aside from different methods, \gls{kpca} has different kernels that would be interesting to explore. Section 1 shows that the choice of the kernel has a significant impact on the final accuracy. It would be interesting to explore the different kernels available.

Lastly, having access to a more powerful computer would allow us to use the nonlinear dimensionality reductions of the report on the full \gls{mnist} dataset and measure the memory usage to more accurately compare the methods' computational requirements.

Based on the discussion in chapter \ref{cha:discussion}, the results from chapter \ref{cha:results} and the problem statement in section \ref{sec:problem-statement}, we conclude:


In this project, we aimed to explore the impact of dimensionality reduction on the performance of a \gls{ml} for image classification and recognition. We focused on comparing linear and nonlinear dimensionality reduction techniques on \gls{mnist}, using a \gls{svm} as our model of choice. Our results showed that, for sample sizes greater than 2000, linear dimensionality reduction techniques were generally faster than nonlinear methods, while achieving similar levels of accuracy. This supports our hypothesis "nonlinear dimensionality reduction methods work as well as linear methods on the \gls{mnist} dataset". However, our results also indicated that nonlinear methods were more robust, in that they could remove more components before a significant drop in accuracy occurred. Overall, our findings support our hypothesis and provide valuable insights into the relative effectiveness of different dimensionality reduction techniques for image classification tasks.

In conclusion, our project has provided valuable insights into the use of dimensionality reduction techniques in image classification and recognition tasks. We have shown that, while nonlinear methods may be more computationally expensive, they can be equally as effective as linear methods in terms of performance. Our findings have important implications for the field of computer vision and can inform future research in this area.

% This chapter contains the concluding remarks of the project. It is based on the discussion in chapter \ref{cha:discussion}, the results from chapter \ref{cha:results} and the problem statement in section \ref{sec:problem-statement}. The chapter concludes with a reflection and perspectives for future work.
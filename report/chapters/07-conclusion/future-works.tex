\chapter{Future works}\label{sec:future-works}
With the insights gained from the project, we have determined several points of interest worth looking into in the future.

Initially, the project included plans for data augmentation in the pipeline, such as slight rotation of the images or artificial errors by whitening pixels of the numbers, making this a prime candidate for future work. These augmentations would artificially increase the size of the dataset, allowing for more splits, more model accuracy, and possibly making the data more nonlinear.

The \gls{mnist} dataset is relatively simple, so it may be interesting to explore more complex datasets and study the impact of dimensionality reduction on those. Some potential datasets already mentioned in the report are the \gls{cifar} and \gls{fashion-mnist} datasets, which may increase the potential of the nonlinear approaches. Alternatively, other data types, for example, audio, would be interesting to look at because it is very different from \gls{mnist}.

Different dimensionality reduction methods would also be interesting. Especially methods not covered in depth for this report, such as \gls{nmf}, would be exciting.

Aside from different methods, \gls{kpca} has different kernels that would be interesting to explore. Section 1 shows that the choice of the kernel has a significant impact on the final accuracy. It would be interesting to explore the different kernels available.

Lastly, having access to a more powerful computer would allow us to use the nonlinear dimensionality reductions of the report on the full \gls{mnist} dataset and measure the memory usage to more accurately compare the methods' computational requirements.

\section{Effect of hyperparameters on the hypothesis}\label{sec:discussion-hyperparams-hypothesis}
The choice of hyperparameters could contribute to the hypothesis. Taking \gls{kpca} as an example, during cross-validation, the scores for the different combinations widely varied, from the worst being 2\%, to the best being 91.5\%.


In Experiment~\ref{sec:experiment-3} it was shown that the choice of \gls{kpca} kernel could severely impact the performance of the confusion matrices that were generated. The experiment used the best configurations for the given kernels found in Experiment~\ref{sec:experiment-1}. The CSV file containing information regarding the different combinations can be found on the project's repository \href{https://github.com/AAU-Dat/P5-Nonlinear-Dimensionality-Reduction/blob/main/src/results/experiment_one/cross_validation_kernel_pca_svm_15000%20(1).csv}{(here)}. The difference in accuracy between \gls{kpca-r} and \gls{kpca-s} was about 2\%. Looking at the Table \ref{tab:error-percentage-pca-kpca-s-kpca-r}, the difference between the kernels was more visible, as the difference in the percentage of numbers that were wrongly predicted between the numbers varied from 0,2\% to 5,6\%. 


The different versions of the \gls{kpca} also used slightly different hyperparameters, contributing to the difference in accuracy. The best \gls{kpca-s}, for example, used $\gamma$=0.001, but the best \gls{kpca-r} used $\gamma$=0.01, and achieved 89.5\% accuracy. If only the kernels were changed, then the $\gamma$ would make a difference. From the CSV file, \gls{kpca-r} with $\gamma$=0.01 achieved an accuracy of 56\%, and \gls{kpca-s} with $\gamma$=0.001 achieved 91.2\%. Thus, from the experiments, hyperparameters could impact the hypothesis.


The choice of hyperparameters could have been made better. According to \gls{sklearn}, other kernels could have been used~\cite{scikit-learn}. Furthermore, the number of components could have been expanded on, and used more different values for \gls{kpca}'s $\gamma$ value. Other methods also use different hyperparameters, which need to be considered. The results regarding the methods with other hyperparameters are unknown, and therefore it cannot be stated whether they will impact on the hypothesis. In future works, experiments regarding the effect of these hyperparameters could be done in more detail to research the effect of the hyperparameters on the hypothesis.

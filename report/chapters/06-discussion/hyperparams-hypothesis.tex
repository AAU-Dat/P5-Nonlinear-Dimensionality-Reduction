
\section{Effect of hyperparameters on the hypothesis}
Another cause that could contribute to the hypothesis is the choice of hyperparameters. Taking \gls{kpca} as an example, during cross-validation, the scores for the different combinations widely varied, from the worst being 2\%, to the best being 91\%.


In experiment three, from the results chapter, it was shown that the choice of \gls{kpca} kernel could severely impact the performance of the confusion matrices that were generated. There, the best configuration for each kernel was chosen, information which is available on the project's repository \href{https://github.com/AAU-Dat/P5-Nonlinear-Dimensionality-Reduction/blob/main/src/results/experiment_one/cross_validation_kernel_pca_svm_15000%20(1).csv}{(here)}. The difference in accuracy was about 2\%. Looking at the Table \ref{tab:error-percentage-pca-kpca-s-kpca-r}, the difference between the kernels was more visible, as the difference in error percentage between the number that was correctly predicted varied from 0,2\% to 5,6\%. 


The different versions of the \gls{kpca} also used slightly different hyperparameters, which contributed to the difference in accuracy. As an example, the best \gls{kpca-s} used $\gamma$=0.001, and achieved 91.5\%, but the best \gls{kpca-r} used $\gamma$=0.01, and achieved 89.5\%. The difference in accuracy was about 2\%. If only the kernels were changed, then the $\gamma$ would make a difference. From the csv file, \gls{kpca-r} with $\gamma$=0.01 achieved an accuracy of 56\%, and \gls{kpca-s} with $\gamma$=0.001 achieved 91.2\%. This fact, and that the accuracy can vary that much could contribute to the hypothesis.


It can be argued that the choice of hyperparameters could have been made better. According to \gls{sklearn}, there are also another kernels that could have been used~\cite{scikit-learn}. Furthermore, the number of components could have been expanded on, as well as using more different values for \gls{kpca}'s $\gamma$ value. There are also other methods, which use different hyperparameters, so the issue of the effect of hyperparameters on the hypothesis could also be applied to them. The results regarding the methods with other hyperparameters are not known, and therefore it cannot be stated whether they will have an impact on the hypothesis. In future works, experiments regarding the effect of these hyperparameters could be done in more detail to research the effect of the hyperparameters on the hypothesis.

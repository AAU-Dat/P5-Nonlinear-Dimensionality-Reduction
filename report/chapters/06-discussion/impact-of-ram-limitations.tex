\section{Impact of memory limitations}\label{sec:impact_of_memory_limitations}
This section will discuss the limitations of memory and the impact of the limitations on the models, with results and other factors in mind. It will be discussed whether or not the impact of limited memory was significant or if it was minor and could be ignored. If the impact was significant, what could be done to improve the situation. 


\subsection{Impact on the results}\label{subsec:impact_on_results}
As stated throughout the project, a limiting factor when running the non-linear models was the memory of available machines. This meant that it was impossible to properly run the model with the entire dataset when reducing the data with a non-linear method; instead, only a smaller section of the dataset was used. The linear methods were not affected by this limitation, as they could be run on the entire dataset.

Using a smaller dataset could have an impact on the results of the final model when using non-linear methods, but through the experiments done, it was noticed that the general trend of the models did not seem to change much after a certain number of samples. Experiment four covered the effect different sample sizes had on the model's accuracy. \autoref{fig:experiment_4_performance_size} shows how at low sample sizes, the model with best accuracy can vary, but as the sample size increases, they all seem to fall into order. This would indicate that the different dimensionality methods are not significantly affected by the limited sample size, after a certain amount, and that the results should still be valid.

If however it was found that the results were significantly affected by the limited sample size, it might be considered finding a smaller dataset, use different methods to reduce the dimensionality, or try to gain access to a more powerfull machine that could handle the larger datasets, with non-linear methods.


\subsection{Impact on time to fit the models}\label{subsec:impact_on_time_to_fit_the_models}
The time to fit the models was also affected by memory limitations. The models could use multiple CPU cores, but this increased memory usage. Since non-linear methods already used much memory, it was only sometimes possible to run the models on multiple cores. An example of how much faster it was with multiple cores was when running \gls{isomap}, which took about 3+ days to run on a machine with only eight GB of memory, meaning that there was not enough ram to run on multiple cores. Whereas for a machine with 32 GB of memory, it took about one day to run on three cores. This shows the impact of limited memory on time to fit the models. The CPU's power also had an effect, but the general still shows that using multiple cores significantly reduced the time to fit the models.

Had time been a significant factor for the project other than comparing the different methods, it would have been considered to run the models on a machine with more memory. A machine with more memory could also have allowed running more models on the same machine, which would have been beneficial for comparing the different methods. The machines used for this project did not have the same CPU power. Therefore it could have been beneficial to use a single powerful machine to run all the models. A single powerful machine would have reduced the time to fit the models and allowed more precise comparisons between the different methods.

The impact of limited memory was significant, but it was not a major factor in the project. As the results were not significantly affected by the limited sample size. Instead, it was generally the time it took to fit the model that could be reduced by having access to more memory. By having multiple machines to run the models, the time impact was shared, had there only been a single machine available that had limited memory, the time impact could have been much more significant.


% Do we want to talk about cores as it is related to time, and uses more memory?
% Has is had an effect on our results that we ran on the non-linear models on a smaller dataset?
% How large of a con for ISOMap is it that the method can not be run without larger amounts of ram?
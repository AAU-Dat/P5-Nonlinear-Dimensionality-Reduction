\section{What we learned}\label{sec:what-we-learned}
To wrap the discussion up, the group will present the insights they have gained from the project.

Over the course of this project, we learned several important lessons about data analysis and \gls{ml}. Practically, we learned how to construct and develop a comprehensive \gls{ml} pipeline for data preprocessing, hyperparameter optimization, cross-validation, and evaluation. This allowed us to effectively train and evaluate our models on various datasets, and taught us how to do so in a systematic and efficient manner in future projects.

Second, we gained a deeper understanding of various dimensionality reduction techniques and their applications. We explored the differences between linear and nonlinear dimensionality reduction methods, and found a strong case for the use of linear methods in the case of \gls{mnist}. We witnessed the impacts of feature engineering on the performance and training time of our models, and learned how dimensionality reduction can be used to mitigate the curse of dimensionality or present a tradeoff between training time and computation requirements with little loss in accuracy. While we did not find an actual usecase for nonlinear dimensionality reduction on our data, the examples presented in \autoref{sec:examples-methods} shows that there at least exists a theoretical use.

Third, we gained experience with the \gls{svm} classification algorithm and its applications. We learned how to use classification algorithms to make predictions on our data and evaluate their performance.

Finally, we learned how to use various tools and techniques to reduce the dimensionality of our data and improve the performance of our models. We explored the use of \gls{pca}, \gls{kpca}, \gls{lda}, and \gls{isomap} for dimensionality reduction, and learned about the strengths and weaknesses of each method.

Overall, this project has given us valuable experience in data analysis and \gls{ml}, and has provided us with a solid first step for further study and research in these fields. It has also provided a good introduction to working theoretically in a scientific setting.

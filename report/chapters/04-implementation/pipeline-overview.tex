\section{Pipeline overview}\label{sec:pipeline-overview}
The development for this project is an extension of the general pipeline from \autoref{fig:basic-machine-learning-pipeline} extended with the considerations from \autoref{cha:theory}.

\autoref{fig:pipeline-overview} shows an overview of the developed pipeline, which consists of five segments: The data sets, the pre-processing, the dimensionality reduction, the machine learning loop, and the output.


\begin{figure}[b!]
    \centering
    \begin{tikzpicture}[node distance=1.5cm, auto, every node/.style={scale=0.7}]
    %MNIST Original logo
    \node (db-slice1) [cylinder, shape border rotate=90, draw, minimum height=1cm,minimum width=3.32cm, aspect=1.1] {};
    \node (db-slice2) [cylinder, shape border rotate=90, draw, minimum height=1cm,minimum width=3.32cm, text centered, above=0pt of db-slice1.before top, anchor=after bottom,aspect=0.1] {MNIST original};
    \node (db-slice3) [cylinder, shape border rotate=90, draw, minimum height=1cm,minimum width=3.32cm, above=0pt of db-slice2.before top, anchor=after bottom, aspect=1.1] {};

    %MNIST Argumented logo
    \node (db-slice4) [cylinder, shape border rotate=90, draw, minimum height=1cm,minimum width=3.7cm, aspect=1.4, shift={($(db-slice1.south)+(0cm,-4.5cm)$)}] {};
    \node (db-slice5) [cylinder, shape border rotate=90, draw, minimum height=1cm,minimum width=3.7cm, text centered, above=0pt of db-slice4.before top, anchor=after bottom,aspect=0.1] {MNIST augmented};
    \node (db-slice6) [cylinder, shape border rotate=90, draw, minimum height=1cm,minimum width=3.7cm, above=0pt of db-slice5.before top, anchor=after bottom, aspect=1.4] {};

    \node (a) [cluster, shift={($(db-slice2.east)+(2cm,-2.5cm)$)}] {pre-processing};
    \node (b) [cluster, shift={($(a.east)+(4cm,0cm)$)}] {No dimensionality reduction};
    \node (c) [cluster, below of=b] {Dimensionality reduction};
    \node (d) [circle, shift={($(c.south west)+(0.5cm,-2.5cm)$)}] {Linear};
    \node (e) [circle, shift={($(c.south east)+(-0.5cm,-2.5cm)$)}] {Non linear};
    \node (f) [cluster, shift={($(b.east)+(3.5cm,0cm)$)}] {Machine learning model};
    \node (g) [cluster, below of=f] {output};

    \node (o) [shift={($(a.west)+(-2cm,0cm)$)}] {};
    \node (p) [shift={($(a.east)+(0.5cm,0cm)$)}] {};
    \node (q) [shift={($(b.east)+(0.5cm,0cm)$)}] {};

    \draw [arrow, -] (db-slice1.south) -- ++(0,0) -| (o.west);
    \draw [arrow, -] (db-slice6.north) -- ++(0,0) -| (o.west);
    \draw [arrow, ->] (o.west) -- (a.west);

    \draw [arrow, ->] (a.east) -- (b.west);
    \draw [arrow, ->] (p.east) -- ++(0,-1) |- (c.west);

    \draw [arrow, -] (c.south) -- ++(0,-0.75) -| (e.north);
    \draw [arrow, -] (c.south) -- ++(0,-0.75) -| (d.north);

    \draw [arrow, ->] (b.east) -- (f.west);
    \draw[arrow, ->] (f) to[loop above,looseness=9] node[above,scale=0.8, xshift=-10,yshift=7] {We train the model with the training data} (f);
    \draw [arrow, -] (c.east) -- ++(0,0) -| (q.east);
    \draw [arrow, ->] (f.south) -- (g.north);
\end{tikzpicture}
    \caption{Overview of the pipeline used for the project.}
    \label{fig:pipeline-overview}
\end{figure}


The first segment, the data sets, covers downloading the dataset, loading it into the program, and augmenting the original dataset to create a new one.

The second segment, pre-processing, covers reshaping the data into a form usable with the \gls{sklearn} library and scaling the data in preparation for the next segment.

The third segment, and the focus of this project, is dimensionality reduction. This segment consists of the four dimensionality reduction algorithms: \gls{pca}, \gls{lda}, \gls{isomap}, and \gls{kpca}, as well as no dimensionality reduction.

The fourth segment is the tuning loop, where grid search with cross-validation determines the best hyperparameters for the \gls{svm} based on the f1-score. Because the project focuses on dimensionality reduction, this loop includes the number of components for dimensionality as a hyperparameter as well.

Finally, the fifth segment is the output. Once the best hyperparameters are determined, the \gls{svm} is trained on the whole dataset and tested on the test set. The results of the grid search - scores, parameters, and scoring time - and the results of the final \gls{svm} - confusion matrix and classification report - are saved.

% \subsection{Data augmentation}

% \begin{figure}[htb!]
%     \centering
%     \begin{tikzpicture}
    \node (db-slice4) [cylinder, shape border rotate=90, draw, minimum height=1cm,minimum width=3.32cm, aspect=1.6] {};
    \node (db-slice5) [cylinder, shape border rotate=90, draw, minimum height=1cm,minimum width=3.32cm, text centered, above=0pt of db-slice4.before top, anchor=after bottom,aspect=0.27] {MNIST};
    \node (db-slice6) [cylinder, shape border rotate=90, draw, minimum height=1cm,minimum width=3.32cm, above=0pt of db-slice5.before top, anchor=after bottom, aspect=1.6] {};

    \node (a) [cluster, shift={($(db-slice5.east)+(3cm,0)$)}] {Data augmentation};


    \node (db-slice1) [cylinder, shape border rotate=90, draw, minimum height=1cm,minimum width=3.7cm, aspect=1.4, shift={($(a.east)+(3cm,-0.7cm)$)}] {};
    \node (db-slice2) [cylinder, shape border rotate=90, draw, minimum height=1cm,minimum width=3.7cm, text centered, above=0pt of db-slice1.before top, anchor=after bottom,aspect=0.1] {MNIST augmented};
    \node (db-slice3) [cylinder, shape border rotate=90, draw, minimum height=1cm,minimum width=3.7cm, above=0pt of db-slice2.before top, anchor=after bottom, aspect=1.4] {};


    \draw [arrow, ->] (db-slice5.east) -- (a.west);
    \draw [arrow, ->] (a.east) -- (db-slice2.west);
\end{tikzpicture}
%     \caption{Data augmentation pipeline segment.}
% \end{figure}
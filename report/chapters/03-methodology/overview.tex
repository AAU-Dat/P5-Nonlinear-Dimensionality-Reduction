

\section{Overview}\label{sec:overview}
What we explore is the impact of especially different augmentations of the original data on the application of different dimensionality reduction techniques, and how this affects the performance of the models.


\begin{description}
    \setlength\itemsep{0em}
    \item[dataset] mnist
    \item[pre-preprocessing] normalization, data augmentation
    \item[preprocessing] linear and non linear dimensionality reduction (PCA, LDA and Kernel PCA)
    \item[models] logistic regression and CNN
    \item[evaluation] accuracy, precision, recall, f1-score, speed/run time, memory usage, and model size
\end{description}


\subsection{Data collection}\label{subsec:data-collection}
The data used is the \gls{mnist} database, which is a collection of handwritten digits. The data is split into a training set of 60,000 images and a test set of 10,000 images. The images are 28x28 pixels, and each pixel is represented by a value between 0 and 255, where 0 is black and 255 is white. The images are grayscale, and the values are the intensity of the pixel. The images are labeled with the digit they represent, and the labels are integers between 0 and 9.

\subsection{Data pre-preprocessing}\label{subsec:data-pre-preprocessing}
The data is normalized by dividing each pixel value by 255, so that the values are between 0 and 1. The data is then augmented by rotating the images by 90, 180 and 270 degrees, and flipping the images horizontally and vertically. This results in 10 times as much data as the original \gls{mnist} dataset.\supervisor{This is just placeholder, but something like this is what we want to do.}

\subsection{Data preprocessing}\label{subsec:data-preprocessing}
The data is then preprocessed by applying linear and non-linear dimensionality reduction techniques. The linear dimensionality reduction techniques are \gls{pca} and \gls{lda}. The non-linear dimensionality reduction technique is \gls{kpca}. The data is reduced to 2 dimensions, so that it can be visualized.







An automation pipeline is created based on Figure~\ref{fig:python-pipeline-model}.


\begin{figure}[htb!]
    \centering
    \includegraphics[width=\textwidth]{figures/pipeline-draft.png}
    \caption{Python project pipeline model.}
    \label{fig:python-pipeline-model}
\end{figure}


\urgent[inline]{write about the pipeline}

\pdfbookmark[0]{English title page}{label:titlepage_en}
\aautitlepage{%
  \englishprojectinfo{
    Linear versus Non-linear Dimensionality Reduction %title
  }{%
    Theoretical data analysis and modeling %theme
  }{%
    Fall Semester 2022 %project period
  }{%
    cs-22-dat-5-05 % project group
  }{%
    %list of group members
    Daniel Runge Petersen\\
    Gustav Svante Graversen\\
    Lars Emanuel Hansen\\
    Raymond Kacso\\
    Sebastian Aaholm
  }{%
    %list of supervisors
    Alexander Leguizamon Robayo
  }{%
    1 % number of printed copies
  }{%
    \today % date of completion
  }%
}{%department and address
  \textbf{Electronics and IT}\\
  Aalborg University\\
  \href{http://www.aau.dk}{http://www.aau.dk}
}{% the abstract
In recent years, dimensionality reduction has become an increasingly important topic in the field of machine learning. Dimensionality reduction refers to the process of reducing the number of features in a dataset, while retaining as much relevant information as possible. This can be useful for improving the performance of machine learning algorithms, as well as for visualizing and understanding the structure of a dataset. In this paper, we compare two different types of dimensionality reduction techniques: linear and nonlinear. We apply these techniques to a popular dataset for image classification and recognition tasks, the \gls{mnist} dataset, and evaluate the performance of a \gls{svm} model on the reduced datasets. Our results show that, for sample sizes greater than 2000, linear dimensionality reduction techniques are generally faster than nonlinear methods, while achieving similar levels of accuracy. However, nonlinear methods are more robust, in that they can remove more components before a significant drop in accuracy occurs. 
}

% \cleardoublepage
% {\selectlanguage{danish}
% \pdfbookmark[0]{Danish title page}{label:titlepage_da}
% \aautitlepage{%
%   \danishprojectinfo{
%     Rapportens titel %title
%   }{%
%     Semestertema %theme
%   }{%
%     Efterårssemestret 2010 %project period
%   }{%
%     XXX % project group
%   }{%
%     %list of group members
%     Forfatter 1\\ 
%     Forfatter 2\\
%     Forfatter 3
%   }{%
%     %list of supervisors
%     Vejleder 1\\
%     Vejleder 2
%   }{%
%     1 % number of printed copies
%   }{%
%     \today % date of completion
%   }%
% }{%department and address
%   \textbf{Elektronik og IT}\\
%   Aalborg Universitet\\
%   \href{http://www.aau.dk}{http://www.aau.dk}
% }{% the abstract
%   Her er resuméet
% }}
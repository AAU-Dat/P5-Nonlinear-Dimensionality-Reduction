\pdfbookmark[0]{English title page}{label:titlepage_en}
\aautitlepage{%
    \englishprojectinfo{
        Linear versus Nonlinear Dimensionality Reduction %title
    }{%
        Theoretical data analysis and modeling %theme
    }{%
        Fall Semester 2022 %project period
    }{%
        cs-22-dat-5-05 % project group
    }{%
        %list of group members
        Daniel Runge Petersen\\
        Gustav Svante Graversen\\
        Lars Emanuel Hansen\\
        Raymond Kacso\\
        Sebastian Aaholm
    }{%
        %list of supervisors
        Alexander Leguizamon Robayo
    }{%
        1 % number of printed copies
    }{%
        \today % date of completion
    }%
}{%department and address
    \textbf{Electronics and IT}\\
    Aalborg University\\
    \href{http://www.aau.dk}{http://www.aau.dk}
}{% the abstract
    In recent years, dimensionality reduction has become an increasingly important topic in \gls{ml}. Previous research has shown that nonlinear dimensionality reduction methods might not outperform linear methods. In this paper, we use \gls{mnist} to compare the dimensionality reduction methods and evaluate the results based on an \gls{svm} model. The evaluation is based on time, accuracy, f1 score, precision, and recall. The results show that linear methods are faster than nonlinear and achieve similar accuracy levels. In contrast, nonlinear methods are less sensitive to fluctuations in accuracy score when the number of dimensions reduced becomes smaller and smaller. With the current findings, nonlinear methods do not outperform linear methods, though the methods' peculiarities from this project serve as a foundation for future work.

}

% \cleardoublepage
% {\selectlanguage{danish}
% \pdfbookmark[0]{Danish title page}{label:titlepage_da}
% \aautitlepage{%
%   \danishprojectinfo{
%     Rapportens titel %title
%   }{%
%     Semestertema %theme
%   }{%
%     Efterårssemestret 2010 %project period
%   }{%
%     XXX % project group
%   }{%
%     %list of group members
%     Forfatter 1\\ 
%     Forfatter 2\\
%     Forfatter 3
%   }{%
%     %list of supervisors
%     Vejleder 1\\
%     Vejleder 2
%   }{%
%     1 % number of printed copies
%   }{%
%     \today % date of completion
%   }%
% }{%department and address
%   \textbf{Elektronik og IT}\\
%   Aalborg Universitet\\
%   \href{http://www.aau.dk}{http://www.aau.dk}
% }{% the abstract
%   Her er resuméet
% }}